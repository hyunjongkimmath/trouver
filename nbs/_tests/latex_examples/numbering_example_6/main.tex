
% Based on https://arxiv.org/abs/2209.00718

\documentclass[12 pt]{amsart}

\usepackage{hyperref}
\usepackage{etex}
\usepackage[shortlabels]{enumitem}
\usepackage{amsmath}
\usepackage{amsxtra}
\usepackage{amscd}
\usepackage{amsthm}
\usepackage{amsfonts}
\usepackage{amssymb}
\usepackage{eucal}
\usepackage[all]{xy}
\usepackage{graphicx}
\usepackage{tikz-cd}
\usepackage{mathrsfs}
\usepackage{subfiles}
\usepackage{mathpazo}
%\usepackage{euler}
\usepackage[colorinlistoftodos, textsize=tiny]{todonotes}
\setlength{\marginparwidth}{2cm}
\usepackage{morefloats}
\usepackage{pdfpages}
\usepackage{thm-restate}
\usepackage[utf8]{inputenc}
\usepackage{epigraph}
\usepackage{csquotes}
\usepackage[margin=1.5in]{geometry}
\usepackage{adjustbox}
\usepackage{scalerel}
\usepackage{stackengine}
\stackMath
\newcommand\reallywidehat[1]{%
\savestack{\tmpbox}{\stretchto{%
  \scaleto{%
    \scalerel*[\widthof{\ensuremath{#1}}]{\kern-.6pt\bigwedge\kern-.6pt}%
    {\rule[-\textheight/2]{1ex}{\textheight}}%WIDTH-LIMITED BIG WEDGE
  }{\textheight}% 
}{0.5ex}}%
\stackon[1pt]{#1}{\tmpbox}%
}
\parskip 1ex




\graphicspath{ {images/} }

\RequirePackage{color}
\definecolor{myred}{rgb}{0.75,0,0}
\definecolor{mygreen}{rgb}{0,0.5,0}
\definecolor{myblue}{rgb}{0,0,0.65}

\usepackage{color}
\newcommand{\daniel}[1]{{\color{blue} \sf
    $\spadesuit\spadesuit\spadesuit$ DANIEL: [#1]}}
\newcommand{\rboxed}[1]{\boxed{#1}}
%\renewcommand{\rboxed}[1]{#1}
\newcommand{\aaron}[1]{{\color{red} \sf
    $\spadesuit\spadesuit\spadesuit$ AARON: [#1]}}



\usepackage{hyperref}
\hypersetup{citecolor=blue}
\usepackage{tikz}
\usetikzlibrary{matrix,arrows,decorations.pathmorphing}

%\numberwithin{equation}{subsection} 
%\numberwithin{figure}{subsection} 

\theoremstyle{plain}
\newtheorem{theorem}[subsection]{Theorem}
\newtheorem{slogan}[subsection]{Slogan}
\newtheorem{proposition}[subsection]{Proposition}
\newtheorem{lemma}[subsection]{Lemma}
\newtheorem{corollary}[subsection]{Corollary}
\newtheorem{situation}[subsection]{Situation}
\newtheorem{problem}[subsection]{Problem}
\theoremstyle{definition}
\newtheorem{definition}[subsection]{Definition}
\newtheorem{remark}[subsection]{Remark}
\newtheorem{example}[subsection]{Example}
\newtheorem{exercise}[subsection]{Exercise}
\newtheorem{counterexample}[subsection]{Counterexample}
\newtheorem{convention}[subsection]{Convention}
\newtheorem{question}[subsection]{Question}
\newtheorem{conjecture}[subsection]{Conjecture} 
\newtheorem{goal}[subsection]{Goal}
\newtheorem{warn}[subsection]{Warning}
\newtheorem{fact}[subsection]{Fact}
\theoremstyle{remark}
\newtheorem{notation}[subsection]{Notation}
\newtheorem{construction}[subsection]{Construction}
\numberwithin{equation}{section}
\newcommand\nc{\newcommand}
\nc\on{\operatorname}
\nc\renc{\renewcommand}
\newcommand\se{\section}
\newcommand\ssec{\subsection}
\newcommand\sssec{\subsubsection}


  
\newcommand\ba{\mathbb A}
\newcommand\bb{\mathbb B}
\newcommand\bc{\mathbb C}
\newcommand\bd{\mathbb D}
\newcommand\be{\mathbb E}
\newcommand\bbf{\mathbb F}
\newcommand\bg{\mathbb G}
\newcommand\bh{\mathbb H}
\newcommand\bi{\mathbb I}
\newcommand\bj{\mathbb J}
\newcommand\bl{\mathbb L}
\newcommand\bm{\mathbb M}
\newcommand\bn{\mathbb N}
\newcommand\bo{\mathbb O}
\newcommand\bp{\mathbb P}
\newcommand\bq{\mathbb Q}
\newcommand\br{\mathbb R}
\newcommand\bs{\mathbb S}
\newcommand\bt{\mathbb T}
\newcommand\bu{\mathbb U}
\newcommand\bv{\mathbb V}
\newcommand\bw{\mathbb W}
\newcommand\bx{\mathbb X}
\newcommand\by{\mathbb Y}
\newcommand\bz{\mathbb Z}

\newcommand\fp{{\mathbb F_p}}
\newcommand\fq{{\mathbb F_q}}

\newcommand\fa{{\mathfrak a}}
\newcommand\fm{{\mathfrak m}}
\newcommand\fg{{\mathfrak g}}
\newcommand\fh{{\mathfrak h}}
\newcommand\fgl{{\mathfrak {gl}}}
\newcommand\fsu{{\mathfrak {su}}}
\newcommand\fsl{{\mathfrak {sl}}}
\newcommand\fso{{\mathfrak {so}}}
\newcommand\fo{{\mathfrak {o}}}
\newcommand\fu{{\mathfrak {u}}}
\newcommand\fsp{{\mathfrak {sp}}}
\newcommand\fgsp{{\mathfrak {gsp}}}



\newcommand\sca{\mathscr A}
\newcommand\scb{\mathscr B}
\newcommand\scc{\mathscr C}
\newcommand\scd{\mathscr D}
\newcommand\sce{\mathscr E}
\newcommand\scf{\mathscr F}
\newcommand\scg{\mathscr G}
\newcommand\sch{\mathscr H}
\newcommand\sci{\mathscr I}
\newcommand\scj{\mathscr J}
\newcommand\sck{\mathscr K}
\newcommand\scl{\mathscr L}
\newcommand\scm{\mathscr M}
\newcommand\scn{\mathscr N}
\newcommand\sco{\mathscr O}
\newcommand\scp{\mathscr P}
\newcommand\scq{\mathscr Q}
\newcommand\scs{\mathscr S}
\newcommand\sct{\mathscr T}
\newcommand\scu{\mathscr U}
\newcommand\scv{\mathscr V}
\newcommand\scw{\mathscr W}
\newcommand\scx{\mathscr X}
\newcommand\scy{\mathscr Y}
\newcommand\scz{\mathscr Z}

\newcommand \ra{\rightarrow}
\newcommand \xra{\xrightarrow}
\newcommand \la{\rightarrow}
\newcommand \xla{\xrightarrow}
\DeclareMathOperator\spec{\text{Spec}}
\DeclareMathOperator\proj{\text{Proj}}
\DeclareMathOperator\rspec{\textit{Spec }}
\DeclareMathOperator\rproj{\textit{Proj }}
\newcommand*{\shom}{\mathscr{H}\kern -.5pt om}
\newcommand*{\stor}{\mathscr{T}\kern -.5pt or}
\newcommand*{\sext}{\mathscr{E}\kern -.5pt xt}
\newcommand \mg{{\mathscr M_g}}

\makeatletter
\providecommand\@dotsep{5}
\renewcommand{\listoftodos}[1][\@todonotes@todolistname]{%
\@starttoc{tdo}{#1}}
\makeatother

\makeatletter
\newcommand{\customlabel}[2]{\protected@write \@auxout {}{\string \newlabel {#1}{{#2}{\thepage}{#2}{#1}{}} }\hypertarget{#1}{#2}}
%\newcommand\sh[1]{\sco_{#1}^{\rm{sh}}}
\newcommand\bundle[1]{{\mathscr E}_{#1}}
\newcommand\fbundle[1]{E_{#1}}
\DeclareMathOperator\id{id}
\DeclareMathOperator\sh{sh}
\DeclareMathOperator\tor{Tor}
\DeclareMathOperator\ext{Ext}
\DeclareMathOperator\red{red}
\renewcommand\hom{\mathrm{Hom}}
\DeclareMathOperator\coker{coker}
\DeclareMathOperator\ord{ord}
\DeclareMathOperator\hilb{Hilb}
\DeclareMathOperator\conf{Conf}
\DeclareMathOperator\rk{rk}
\DeclareMathOperator\di{div}
\DeclareMathOperator\pic{Pic}
\DeclareMathOperator\lcm{lcm}
\DeclareMathOperator\rank{rank}
\DeclareMathOperator\codim{codim}
\DeclareMathOperator\vol{Vol}
\DeclareMathOperator\supp{Supp}
\DeclareMathOperator\spn{Span}
\DeclareMathOperator\im{im}
\DeclareMathOperator\End{End}
\DeclareMathOperator\sym{Sym}
\DeclareMathOperator\pgl{PGL}
\DeclareMathOperator\sat{Sat}
\DeclareMathOperator\blow{Bl}
\renewcommand\sp{\mathrm{Sp}}
\DeclareMathOperator\gsp{GSp}
\DeclareMathOperator\sgn{sgn}
\DeclareMathOperator\triv{triv}
\DeclareMathOperator\ab{ab}
\DeclareMathOperator\std{std}
\DeclareMathOperator\perm{perm}
\DeclareMathOperator\gal{Gal}
\DeclareMathOperator\tr{tr}
\DeclareMathOperator\frob{Frob}
\DeclareMathOperator\chr{\text{char}}
\newcommand\bk{{\Bbbk}}
\newcommand\ul{\underline}
\newcommand\ol{\overline}
\DeclareMathOperator\pr{pr}
\DeclareMathOperator\ev{ev}
\DeclareMathOperator\cl{Cl}
\DeclareMathOperator\res{Res}
\DeclareMathOperator\prob{Prob}
\DeclareMathOperator\Mod{Mod}
\DeclareMathOperator\HMod{HMod}
\DeclareMathOperator\I{I}
\DeclareMathOperator\PW{PW}
\DeclareMathOperator\HPW{HPW}
\DeclareMathOperator\maj{maj}
\DeclareMathOperator\inv{inv}
\DeclareMathOperator\isom{isom}
\DeclareMathOperator\mor{mor}
\DeclareMathOperator\aut{Aut}
\DeclareMathOperator\gl{GL}
\DeclareMathOperator\spin{Spin}
\DeclareMathOperator\pin{Pin}
\renewcommand\sl{\mathrm{SL}}
\DeclareMathOperator\asl{\mathrm{ASL}}
\DeclareMathOperator\mat{Mat}
\DeclareMathOperator\stab{Stab}
\DeclareMathOperator\so{SO}
\DeclareMathOperator\su{SU}
\renewcommand\u{\mathrm{U}}
\DeclareMathOperator\lie{Lie}
\DeclareMathOperator\ad{ad}
\DeclareMathOperator\Ad{Ad}
\renewcommand\o{{\rm{O}}}
\newcommand\osp{{\rm{O^*}}}
\DeclareMathOperator\sel{Sel}
\DeclareMathOperator\Br{Br}
\DeclareMathOperator\val{val}
\DeclareMathOperator\quo{Quo}
\DeclareMathOperator\diag{diag}
\DeclareMathOperator\out{Out}
\DeclareMathOperator\orbits{Orbits}
\DeclareMathOperator\sm{sm}
\DeclareMathOperator\shf{shf}
\DeclareMathOperator\disc{disc}
\DeclareMathOperator\average{Average}
\DeclareMathOperator\spinor{sp}
\DeclareMathOperator\grp{grp}
\DeclareMathOperator\gr{Gr}
\DeclareMathOperator\et{\acute et}
\DeclareMathOperator\tame{tame}
\DeclareMathOperator\sing{sing}
\DeclareMathOperator\an{an}
\DeclareMathOperator\jac{Jac}
\DeclareMathOperator\colim{colim}

\renewcommand\top{\mathrm{top}}
\DeclareFontFamily{U}{wncy}{}
\DeclareFontShape{U}{wncy}{m}{n}{<->wncyr10}{}
\DeclareSymbolFont{mcy}{U}{wncy}{m}{n}
\DeclareMathSymbol{\Sha}{\mathord}{mcy}{"58} 

\renewcommand{\sectionautorefname}{\S$\!$}
\renewcommand{\subsectionautorefname}{\S$\!$}
\renewcommand{\subsubsectionautorefname}{\S$\!$}


\setcounter{MaxMatrixCols}{20}

\def\listtodoname{List of Todos}
\def\listoftodos{\@starttoc{tdo}\listtodoname}

\title[Applications around the Putman-Wieland conjecture]{Applications of the algebraic geometry of the Putman-Wieland conjecture}
\author{Aaron Landesman and Daniel Litt}

\usepackage{microtype}
\begin{document}



\section{Isotypicity and non-unitarity}\label{section:bilinear-pairing}
The main result of this section is \autoref{theorem:isotypic}, which states that
counterexamples to Putman-Wieland in genus $\geq 3$ cannot be isotypic, i.e.,
there exists an element of 
$H^1(\Sigma_{g'}, \mathbb C)^\rho$ with infinite orbit under the action of a finite
index subgroup of the mapping class group. We show more, namely \autoref{theorem:non-unitary}: if $X\to Y$ is an $H$-cover, where $Y$ has genus at least $3$, the virtual action of the mapping class group of $Y$ on an $H$-isotypic component of the cohomology of $X$ is non-unitary.

In \autoref{corollary:boggi-looijenga} we use this to show how a
 result from the retracted paper of Boggi-Looijenga \cite{boggiL:curves-with-prescribed-symmetry} would imply the Putman-Wieland conjecture.

Our main tool for proving this is a natural bilinear pairing, which we
next introduce.
Let $C$ be a smooth proper connected curve of genus $g$, $D\subset C$ a reduced
divisor, $E_\star$
a parabolic vector bundle on $(C,D)$, and $E := E_0$. 
As described in \cite[(4.5)]{landesmanL:canonical-representations}, there is a bilinear pairing
\begin{align}
	\label{equation:bilinear-pairing}
	B_E:
(E \otimes \omega_C(D))\times (E^\vee\otimes \omega_C) \to
\omega_C^{\otimes 2}(D)
\end{align}
given as the composition $$B_E: (E \otimes \omega_C(D))\times (E^\vee\otimes \omega_C)\overset{\otimes}{\longrightarrow} (E\otimes E^\vee)\otimes \omega_C^{\otimes 2}(D)\overset{\on{tr}\otimes \on{id}}{\longrightarrow} \omega_C^{\otimes 2}(D),$$
where $\on{tr}$ denotes the trace pairing ${E}\otimes {E}^\vee\to
\mathscr{O}_C.$ 
By restriction to $H^0(C, \widehat{E}_0 \otimes \omega_C(D)) \subset H^0(C, E \otimes
\omega_C(D))$, we also obtain an induced pairing
\begin{align*}
	H^0(C, \widehat{E}_0 \otimes \omega_C(D)) \times H^0(C, E^\vee \otimes
	\omega_C) \to H^0(C, \omega^{\otimes 2}_C(D)).
\end{align*}

%\begin{proposition}\label{proposition:pairing-result}
%	Let $E_\star$ be a semistable parabolic bundle on $(C,D)$ of parabolic
%	degree zero, with underlying vector bundle $E$. 
%	Suppose $S \subset E \otimes \omega_C(D)$ is a subbundle of rank $c$ with sections
%	$s_1, \ldots, s_c: \mathscr O \to E \otimes \omega_C(D)$ so that
%	generically generate $S$, for which the
%	corresponding pairings
%	\begin{align*}
%		H^0(C, B_E(s_i, \bullet)): H^0(C, E^\vee \otimes \omega_C) \to
%		H^0(C, \omega_C^{\otimes 2}(D))
%	\end{align*}
%	induced by $B_E$ of \eqref{equation:bilinear-pairing} on global sections
%	have rank $r_i$.
%	Then $\on{rk}(E)\geq cg-\sum_{i=1}^c r_i.$
%\end{proposition}
%\begin{proof}
%	Let $f_i := E^\vee \otimes \omega_C \to \omega_C^{\otimes 2}(D)$ denote
%	the map adjoint to the section $s_i$
%	and let $U_i := \ker f_i$.
%	By definition of $r_i$ and the long exact sequence on cohomology
%	\begin{equation}
%		\label{equation:}
%		\begin{tikzcd}
%			0 \ar {r} & H^0(C, U_i) \ar {r} & H^0(C, E \otimes
%			\omega_C) \ar {r} & H^0(C, \omega_C^{\otimes 2}(D)) 
%	\end{tikzcd}\end{equation}
%	we find $h^0(C, U_i) = H^0(C, E \otimes \omega_C) - r_i$.
%	For $m \leq c$, a diagram chase yields
%	$h^0(C, \cap_{i=1}^m U_i) \geq H^0(C, \cap_{i=1}^{m-1} U_i) - r_m$.
%	Hence, induction on $m$ implies
%	$h^0(C, \cap_{i=1}^m U_i) \geq h^0(C, E^\vee \otimes \omega_C) - \sum_{i=1}^c r_i$.
%
%	We will conclude by explaining how to apply
%\autoref{proposition:generic-parabolic-global-generation}(II) with $\delta = r$,
%and
%$U = \cap_{i=1}^c U_i$, so that the corank of $U$ is $c$.
%Set $n=\deg D$. 
%Consider the coparabolic bundle $\reallywidehat{(E_\star)^\vee \otimes
%\omega_C(D)}$, which is the Serre dual coparabolic bundle
%to $E_\star$ as in \cite[Proposition 2.6.6]{LL:geometric-local-systems}.
%This is coparabolically stable by definition and
%has slope $\mu(\reallywidehat{(E_\star)^\vee \otimes
%\omega_C(D)}) = 2g - 2+n$,
%since $E_\star$ has parabolic slope $0$.
%
%By unwinding the definition of the Serre dual coparabolic bundle, one may verify $\reallywidehat{(E_\star)^\vee \otimes
%\omega_C(D)}_0=E^\vee \otimes \omega_C.$
%Therefore, the assumption of the proposition implies
%\begin{align*}
%	h^0(C,\reallywidehat{(E_\star)^\vee \otimes
%\omega_C(D)}_0) - h^0(C, U) = 
%h^0(C,E^\vee \otimes \omega_C) - h^0(C, U) = 
%\sum_{i=1}^c r_i.
%\end{align*}
%Applying \autoref{proposition:generic-parabolic-global-generation}(II) 
%to the coparabolic bundle
%$\reallywidehat{(E_\star)^\vee \otimes
%\omega_C(D)}$
%with $U = \cap \ker(f_i)$ as defined above so that $\delta = r$ we find 
%$\rk E=\rk \reallywidehat{(E_\star)^\vee \otimes \omega_C(D)} \geq cg -
%\sum_{i=1}^c r_i.$
%\end{proof}

\begin{theorem}
	\label{theorem:isotypic-derivative}
	Let $E_\star$ be a semistable parabolic bundle on $(C,D)$ of parabolic
	degree zero, with underlying vector bundle $E := E_0$. 
	Suppose $g \geq 2$ and that the pairing $H^0(\widehat{E}_0 \otimes \omega_C(D)) \otimes H^0(E^\vee
	\otimes \omega_C) \to H^0(C, \omega_C^{\otimes 2}(D))$ vanishes.
	Then $g = 2$.
\end{theorem}
\begin{proof}
	First, we may assume $\rk E > 1$ by \cite[Proposition
	4.2.3]{landesmanL:canonical-representations}, as $g \geq 2$.
	If $F_\star$ is a parabolic sheaf, we call the image of $H^0(C, F_0)
	\otimes \mathscr O_C \to F_0 \subset F_\star$ the globally generated
	subsheaf of
	$F_\star$, which is by definition also a subsheaf of $F_0$.
	Let $U \subset \widehat{E}_\star \otimes \omega_C(D)$ denote the globally generated
	subsheaf of $\widehat{E}_\star \otimes \omega_C(D)$ and
	$V$ denote the globally generated subsheaf  
	of $\reallywidehat{((E_\star)^\vee \otimes \omega_C(D))}_0$.
	Under the identifications 
\begin{align*}
	\left(\widehat{E}_\star \otimes \omega_C(D)\right)_0 &= \widehat{E}_0 \otimes \omega_C(D) \\
\reallywidehat{((E_\star)^\vee \otimes \omega_C(D))}_0 &= E^\vee \otimes
\omega_C,
\end{align*}
(see \cite[Definition 2.6.1]{LL:geometric-local-systems} for the notion of a dual of a parabolic bundle)
$U$ and $V$ are also the globally generated subsheaves of
$\widehat{E}_0 \otimes \omega_C(D)$ and $E^\vee \otimes \omega_C$ respectively.

Let $c_V := \rk E - \rk V$ and $c_U := \rk E - \rk U.$
Using \autoref{proposition:generic-parabolic-global-generation}
applied to the bundles
$E_\star \otimes \omega_C(D)$ and
$E_\star^\vee \otimes \omega_C(D)$,
we know $\rk E \geq g c_V$ and $\rk E \geq g c_U$.

On the other hand, we claim $\rk U + \rk V \leq \rk E$.
Granting this claim, we find $c_V + c_U \geq \rk E$.
Since $\rk E \geq g c_V$ and $\rk E \geq g c_{U}$, adding these gives
\begin{align*}
	2 \rk E \geq g(c_V + c_U) \geq g \rk E,
\end{align*}
implying $2 \geq g$.

It remains to show $\rk U + \rk V \leq \rk E$.
We will argue this using the fact that an isotropic subsheaf for a non-degenerate quadratic
form
on a vector bundle of rank $2\rk E$ has rank at most $\rk E$.
Indeed, consider the quadratic form $q_E$ on $E \otimes \omega_C(D) \oplus
E^\vee \otimes \omega_C$ associated to the nondegenerate bilinear form $B_E$ of
\eqref{equation:bilinear-pairing}:
\begin{align*}
	q_E : E \otimes \omega_C(D) \oplus
	E^\vee \otimes \omega_C &\to \omega_C^{\otimes 2}(D) \\
	(v,w) &\mapsto B_E(v,w).
\end{align*}
Any vector bundle subsheaf of $E \otimes \omega_C(D) \oplus E^\vee \otimes \omega_C$ 
isotropic for this quadratic form has rank at most $\rk E$, as may be verified
on the generic fiber using that isotropic subspaces of a rank $2 \rk E$
non-degenerate quadratic space have dimension at most $\rk E$.
Therefore, it is enough to
show $U \oplus V$ is killed under $q_E$.
Using $q_E(U \oplus V) = B_E(U \times V)$,
it is enough to show $B_E(U \times V) = 0$.
We have a commutative diagram
\begin{equation}
	\label{equation:}
	\begin{tikzcd} 
		(H^0(C, U) \otimes \mathscr O_C)\times  (H^0(C, V) \otimes \mathscr O_C) \ar {r} \ar {d}
		& H^0(C, \omega_C^{\otimes 2}(D)) \otimes \mathscr O_C \ar {d} \\
		U \times V \ar {r}{B_E} & \omega_C^{\otimes 2}(D)
\end{tikzcd}\end{equation}
where the top horizontal map vanishes by assumption.
Since the vertical maps are surjective, the bottom horizontal map satisfies $B_E(U \times
V) = 0$, as desired.
\end{proof}

\subsection{Proof of isotypicity}
\label{subsection:isotypic-proof}

We now deduce our desired isotypicity consequence, \autoref{theorem:isotypic} and \autoref{theorem:non-unitary}, 
for the Putman-Wieland conjecture.
With setup as in \autoref{conjecture:putman-wieland-intro},
we have a cover $\Sigma_{g',n'} \to \Sigma_{g,n}$ and 
an action of a finite index subgroup $\Gamma \subset \on{Mod}_{g,n+1}$
on $H^1(\Sigma_{g',n'}, \mathbb C)$. We aim to show that if $\rho$ is any
irreducible $H$ representation
so that every element of the
characteristic subspace
$H^1(\Sigma_{g'}, \mathbb C)^\rho \subset H^1(\Sigma_{g',n'}, \mathbb C)$ has
finite orbit under $\Gamma$,
then $g \leq 2$.

\begin{proof}[Proof of \autoref{theorem:isotypic} and \autoref{theorem:non-unitary}]
We first prove \autoref{theorem:non-unitary}.

	Assume to the contrary that the virtual action of  
the mapping class group on $H^1(\Sigma_{g'}, \mathbb
	C)^\rho$ is unitary. Let 
	$\mathscr{X}^\circ\xrightarrow{\widetilde{f}^\circ}
	\mathscr{C}^\circ\xrightarrow{\pi^\circ} \mathscr{M}$ be a versal
	family of $H$-covers, and $\mathscr{X}\to \mathscr{C}\to \mathscr{M}$ the associated families of proper curves. Let $m\in \mathscr{M}$ be a point, and $X\to Y$ the fiber over $\mathscr{M}$ 
	Let $E^\rho_\star$ be the parabolic bundle on $Y$ corresponding to the
	representation $\rho$ as in \autoref{notation:rep-to-vector-bundle}.
	
	We claim the map
	$\overline\nabla_m^\rho$ vanishes identically, as $W^1R^1\pi^\circ_*\rho$, the variation of Hodge structure on $\mathscr{M}$ associated to $H^1(X, \mathbb
	C)^\rho$, is unitary by assumption. Indeed, by
	\cite[1.13]{deligne1987theoreme}, we may write
	$$W^1R^1\pi^\circ_*\rho=\bigoplus_i \mathbb{V}_i\otimes W_i,$$ where the
	$\mathbb{V}_i$ are complex variations of Hodge structure with
	irreducible monodromy and the $W_i$ are constant variations. The
	$\mathbb{V}_i$ carry a unique structure of a $\mathbb{C}$-VHS, up to
	renumbering. As $W^1R^1\pi^\circ_*\rho$ is unitary by assumption, the
	same is true of each $\mathbb{V}_i$, and so the Hodge filtration of each
	$\mathbb V_i$ has length at most one. This proves the claim that
	$\overline\nabla_m^\rho = 0$.
	
%	By the last part of \autoref{lemma:period-map-kernel},
%	$\overline{\nabla}^\rho_m$ vanishes identically for each $m\in \mathscr{M}$.
Note that $\overline{\nabla}^\rho_m$ is the weight $1$ part of
the map adjoint to the multiplication map
\begin{align*}
	H^0(Y, E^\rho_0 \otimes \omega_Y(D)) \otimes H^0((E^\rho_0)^\vee
	\otimes \omega_Y) \to H^0(Y, \omega_Y^{\otimes 2}(D))
\end{align*}
	by \cite[Theorem 4.1.6]{landesmanL:canonical-representations}.
	Since the 
	subspace of $H^0(Y, E^\rho_0 \otimes \omega_Y(D))$ corresponding to $W^1\cap F^1$
	is $H^0(Y, \widehat{E}_0^\rho \otimes \omega_Y(D))$ by
	\autoref{lemma:weight-1-sections},
	we obtain that the multiplication map
	\begin{align*}
		H^0(Y, \widehat{E}_0^\rho \otimes \omega_Y(D)) \otimes H^0((E^\rho_0)^\vee
	\otimes \omega_Y) \to H^0(Y, \omega_Y^{\otimes 2}(D))
	\end{align*}
	also vanishes.
	Using \autoref{theorem:isotypic-derivative}, this implies $g \leq 2$.
	
	\autoref{theorem:isotypic} is immediate, as representations with finite image are unitary.
\end{proof}

As a consequence, we show how a claimed result from a paper of Boggi-Looijenga (which has since been retracted by the authors) implies the
Putman-Wieland conjecture.

\begin{corollary}
	\label{corollary:boggi-looijenga}
	Suppose \cite[Theorem
	B(i)]{boggiL:curves-with-prescribed-symmetry} were
	true.
	Then the Putman-Wieland conjecture,
	\autoref{conjecture:putman-wieland-intro}, would hold for all
	 $g \geq 3$.
\end{corollary}
\begin{remark}
	\label{remark:}
	We note that, unfortunately, there is a fatal error in the
	proof of \cite[Theorem
	B(i)]{boggiL:curves-with-prescribed-symmetry},
	appearing in \cite[Lemma 1.6]{boggiL:curves-with-prescribed-symmetry}
	(which is used in the proof of 
\cite[Theorem 1.1]{boggiL:curves-with-prescribed-symmetry},
and hence of
\cite[Theorem B(i)]{boggiL:curves-with-prescribed-symmetry}).
The error comes in the penultimate sentence of the proof of \cite[Lemma
1.6]{boggiL:curves-with-prescribed-symmetry},
where is it claimed
that ``It follows\dots,'' but in fact no argument is given for this claim. It is for this reason that the authors of \cite{boggiL:curves-with-prescribed-symmetry} retracted that paper.
\end{remark}


\begin{proof}
	Suppose $g \geq 3$, and we are
	given a finite \'etale $H$-cover of Riemann surfaces
	$f: \Sigma_{g',n'} \to \Sigma_{g,n}$ furnishing a counterexample to
	Putman-Wieland.
	This means some 
	subrepresentation $\chi \subset H^1(X, \mathbb
	C)$ has finite orbit under the action of the mapping class group
	$\on{Mod}_{g,n+1}$.
	The isotypicity statement of 	
	\cite[Theorem B(i)]{boggiL:curves-with-prescribed-symmetry}
	implies that if $\chi \subset H^1(X, \mathbb
	C)$ has finite orbit under the action of the mapping class group
	$\on{Mod}_{g,n+1}$, 
	every element of the $\chi$-isotypic component
	$H^1(X, \mathbb C)^\chi$ has finite orbit under the action of 
	$\on{Mod}_{g,n+1}$.	
	By definition, this means
	$f: \Sigma_{g',n'} \to \Sigma_{g,n}$ is $\chi$-isotypic
	in the sense of \autoref{definition:pw-counterexample},
	which contradicts \autoref{theorem:isotypic}.
\end{proof}

\end{document}

