% This example is based on https://arxiv.org/abs/2210.03035
% In this example, there are enumerate environments, which should not get their
% own `part`, cf. `divide_latex_text` in `16_latex.convert.ipynb`.
\documentclass[10pt]{article}
\usepackage[utf8]{inputenc}
\usepackage[T1]{fontenc}
\usepackage{amsmath}
\usepackage{amsfonts}
\usepackage{amssymb}
\usepackage{mhchem}
\usepackage{stmaryrd}
\usepackage{bbold}
\usepackage{graphicx}
\usepackage[export]{adjustbox}
\graphicspath{ {./images/} }
\usepackage{hyperref}
\hypersetup{colorlinks=true, linkcolor=blue, filecolor=magenta, urlcolor=cyan,}
\urlstyle{same}

\begin{document}

\section{CONTENTS}
\begin{enumerate}
  \item Introduction 2

  \item Preliminaries $\quad 7$

\end{enumerate}
2.1. Categorical preliminaries $\quad 7$

2.2. On the motivic Spanier-Whitehead category and Milnor-Witt K-theory 8

2.3. $\mathbb{A}^{1}$-derived category and $\mathbb{A}^{1}$-homology 9

\begin{enumerate}
  \setcounter{enumi}{3}
  \item $\mathbb{A}^{1}$-Spanier-Whitehead category of cellular smooth schemes 11
\end{enumerate}
3.1. Cellular schemes 12

3.2. Cellular Spanier-Whitehead category 13

\begin{enumerate}
  \setcounter{enumi}{4}
  \item The cellular homology of Morel-Sawant on cellular Thom spaces 14

  \item Spanier-Whitehead cellular complex 18

\end{enumerate}
5.1. Definitions and basic properties 18

5.2. Endomorphisms, traces, and characteristic polynomials 19

5.3. Cellular Grothendieck-Lefschetz Trace Formula 21

\begin{enumerate}
  \setcounter{enumi}{6}
  \item Rationality of the zeta function 22

  \item $\mathbb{A}^{1}$-logarithmic zeta functions and real points 24

  \item Computing $\mathbb{A}^{1}$-logarithmic zeta functions and examples 27

\end{enumerate}
8.1. $\mathbb{A}^{1}$-zeta function of Frobenius endomorphisms using Theorem $6.2 \quad 28$

8.2. $\mathbb{A}^{1}$-logarithmic zeta functions via Hoyois's trace formula 31

8.3. The logarithmic zeta function of non-cellular schemes 34

\begin{enumerate}
  \setcounter{enumi}{9}
  \item Motivic measures 34
\end{enumerate}
9.1. The $\mathbb{A}^{1}$-trace as a motivic measure 35

9.2. The enriched zeta function as a motivic measure 35

References 36

Date: October 3, 2022.

\end{document}