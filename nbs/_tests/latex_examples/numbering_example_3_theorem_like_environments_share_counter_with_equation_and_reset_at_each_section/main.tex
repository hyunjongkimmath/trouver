\documentclass{amsart}
\usepackage[utf8]{inputenc}
\usepackage{amsmath, amsfonts, amssymb, amsthm, amsopn}

\numberwithin{equation}{section}

\theoremstyle{plain}
\newtheorem*{theorem*}{Theorem}
\newtheorem*{theoremA}{Theorem A}
\newtheorem*{theoremB}{Theorem B}
\newtheorem{theorem}[equation]{Theorem}
\newtheorem{proposition}[equation]{Proposition}
\newtheorem{lemma}[equation]{Lemma}
\newtheorem{corollary}[equation]{Corollary}

\theoremstyle{definition}
\newtheorem{definition}[equation]{Definition}
\newtheorem{example}[equation]{Example}
\newtheorem*{acknowledgements}{Acknowledgements}
\newtheorem*{conventions}{Conventions}

\theoremstyle{remark}
\newtheorem{remark}[equation]{Remark}

\begin{document}

\section{Introduction}

\begin{theorem}
This is Theorem 1.1. This is because the \verb|\numberwithin{equation}{section}| makes the section number included in the equation counter and because the \\
\verb|\newtheorem{theorem}[equation]{Theorem}| command makes the environment \verb|theorem| be counted by the equation counter.
\end{theorem}

The following makes an equation labeled 1.2; 
\begin{equation}
5 + 7 = 12
\end{equation}

\begin{theorem*}
This Theorem is unnumbered
\end{theorem*}

\begin{corollary}
This is Corollary 1.3.
\end{corollary}

\section{Another section}
\begin{theorem}
This is theorem 2.1
\end{theorem}

The following is labeled 2.2:
\begin{equation}
3+5 = 8.
\end{equation}

\end{document}